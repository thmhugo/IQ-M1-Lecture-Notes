\documentclass{article}
\usepackage[utf8]{inputenc}
\usepackage{ geometry}
\geometry{legalpaper, margin=1.3in}
\begin{document}
\section{La physique quantique}
Le résultat d'une mesure appartient forcément à un ensemble de résultats propres {a}
A chaque valeur propre $a$ est associé un état propre, c'est a dire une fonction propre
$\psi_a$.
Cette fonction est telle que si $\psi(x, t_0)=\psi_a(x)$, la mesure donnera à coup sûr $a$.
\end{document}